\chapter{Conclusioni}
Lo scopo della tesi è stato il miglioramento del riconoscimento di oggetti su dataset termici. Come si è potuto vedere nell'ultimo capitolo riguardante gli esperimenti si sono prodotti avanzamenti aggiungendo come fossero mattoncini sempre più tecniche, fino a costruire un infrastruttura in grado di effettuare rilevazioni migliori a parità di condizioni difficili ma con immagini convenzionali. 

Il campo di utilizzo di un detector di questo tipo può spaziare dalla videosorveglianza fino ad avere applicazioni nella guida autonoma, passando per un incremento nella sicurezza dei treni, come d'altronde abbiamo potuto vedere tramite i video di \ac{RFI}. L'utilizzo di telecamere termiche, oltre a migliorare la visione in condizioni di scarsa illuminazione permette di vedere, anche oltre la nebbia o una fitta pioggia. 

La parte iniziale del progetto di tesi ha messo alla luce come è possibile ottenere buoni risultati attraverso l'utilizzo di dataset ampi; tuttavia successivamente ci siamo focalizzati sul miglioramento della detection in un sottoinsieme del dataset \ac{kmpd} annotato con le vetture.

Inizialmente è stata fissata una base di partenza data dall'addestramento di RetinaNet (\ref{sec:retinanet}) sul dataset di FLIR (Tabella \ref{table:baseline_car_kaist}). Tramite una operazione detta di \textit{fine tuning} è stato raggiunto un incremento complessivo della \ac{map} su \ac{kmpd} del $25\%$(Tabella \ref{tab:fine_tuning_kaist_flir_increment}).

Considerando i pochi dati a nostra disposizione tra le tecniche utilizzate allo scopo di migliorare ulteriormente la \ac{map} è stata di particolare rilevanza quella di \textit{Data Augmentation}, in particolare \textit{AutoAugment} e \textit{RandAugment}. 
Utilizzando delle politiche predefinite per \textit{AutoAugment} sono stati riscontrati incrementi nella \ac{map} che variano tra il $2\%$ ed il $4\%$. Con \textit{RandAugment} invece è stato possibile effettuare una fase di ottimizzazione degli iperparametri che ha portato ad avere risultati comparabili ad \textit{AutoAugment} (Figura \ref{fig:ra_person}). 

Tramite delle \acf{GAN} è stato generato un dataset \textit{fasullo} partendo da immagini di \ac{kmpd} RGB su cui sono stati effettuati esperimenti. Questi esperimenti hanno evidenziato come quando ci si trova in una situazione di totale assenza di dati le \ac{GAN} possono aiutare nel miglioramento delle performance. 

Sono stati inoltre effettuati esperimenti di \textit{Transfer Learning} in quanto il trasferimento di conoscenze tra due domini differenti ha aiutato nel rilevamento di operai nel dataset di \ac{RFI} nei quali non ci sono annotazioni e tramite un algoritmo basato sull'indice di Jaccard (\ref{sec:iou_over_time}) sono state migliorate le rilevazioni. 


\section{Sviluppi futuri}
Questo progetto di ricerca mette alla luce come è possibile realizzare dei detector basati su immagini termiche per migliorare la \textit{object detection}. Prendendo come spunto questo lavoro si potrebbe realizzare un detector che si basa sia su immagini termiche che convenzionali per realizzare previsioni sempre più accurate.

Un aspetto molto importante è la quantità di dati a disposizione dalla quale dipende l'efficacia di un modello di \textit{Machine Learning}.
Purtroppo la scarsità di dataset utilizzabili per realizzare l'obbiettivo di questa tesi ha giocato a nostro sfavore. Una direzione di ricerca quindi potrebbe essere lo sviluppo di ulteriori insiemi di dati sulla falsariga del dataset \acf{kmpd}. 

Infine abbiamo potuto verificare che la tecnica di \textit{Data Augmentation} che ha portato subito a dei risultati migliori è stata \textit{AutoAugment} nonostante non sia stata realizzata una fase di addestramento specifica per il dataset utilizzato da noi. Si potrebbe quindi, con hardware più potente o avendo a disposizione molto più tempo, portare a termine la generazione di policy specifiche di \textit{AutoAugment}. 