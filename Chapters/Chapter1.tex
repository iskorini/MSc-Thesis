\chapter{Object Detection}
L'Object Detection è un \textit{task} legato al mondo della \textit{computer vision}
che consiste nel rilevare e classificare istanze di oggetti in immagini o video.

Negli ultimi anni, grazie soprattutto all'avvento delle \ac{gpu}, c'è stato un 
incremento notevole del potere computazionale. Questo ha portato a sviluppare tecniche 
sempre più raffinate allo scopo di raggiungere prestazioni sempre migliori. 

Sempre grazie allo sviluppo di hardware sempre più potente l'interesse sta sempre 
più virando verso il mondo del \textit{Deep Learning}. 
In questo capitolo cercheremo di classificare le varie metodologie con cui si 
porta a compimento la \textit{Object Detection}. 
La letteratura sui detector è molto disomogenea e variegata, prenderemo quindi come riferimento i lavori di 
\textit{Licheng Jiao, Fan Zhang, Fang Liu, Shuyuan Yang, Lingling Li, Zhixi Feng and Rong Qu} 
\cite{DBLP:journals/corr/abs-1907-09408} e 
\textit{Zhengxia Zou, Zhenwei Shi, Yuhong Guo and Jieping Ye} \cite{DBLP:journals/corr/abs-1905-05055}.

Una prima, ma importante distinzione va fatta tra metodi non basati sul \textit{deep learning} e metodi basati sul \textit{deep learning}.
\section{Tipologie di detector}

U
\subsection{One Stage Detector}
\label{subsec:one_stage_detector}
\subsection{Two Stage Detector}
\label{subsec:two_stage_detector}

\section{RetinaNet}
\subsection{Focal Loss}