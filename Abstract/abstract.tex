\documentclass[11pt]{article}

\usepackage[italian]{babel}
\usepackage[utf8]{inputenc} 
\usepackage{hyperref}
\usepackage[normalem]{ulem}
\usepackage[font=large, tablename= ]{caption}
\useunder{\uline}{\ul}{}
\renewcommand{\baselinestretch}{1.3}
\setlength{\textwidth}{16 cm}
\setlength{\oddsidemargin}{0 cm}
\setlength{\topmargin}{-1.5 cm}
\setlength{\textheight}{30 cm}
\begin{document}

\begin{table}[]
\centering
\caption*{Tecniche di apprendimento profondo per riconoscimento di oggetti in video termici}
\caption*{Deep Learning Techniques for Object Detection in Thermal Videos}
\begin{tabular}{lll}
\textbf{Candidato:}   & Federico Schipani & \href{mailto:federico.schipani@stud.unifi.it}{\texttt{federico.schipani@stud.unifi.it}}  \\
\textbf{Relatore:}    & Marco Bertini  & \href{mailto:marco.bertini@unifi.it}{\texttt{marco.bertini@unifi.it}}                           \\
\end{tabular}
\end{table}
\subsection*{Abstract}
Il \textit{Machine Learning}, detto anche Apprendimento Automatico, nasce come diramazione dell'intelligenza artificiale e trova  campi di applicazione sempre più ampi nel mondo contemporaneo. Il trend attuale prevede l'utilizzo sempre maggiore di Reti Neurali artificiali, ovvero dei modelli matematici che per molti versi cercano di ricalcare la funzionalità di un cervello umano allo scopo di sostituirlo per quei compiti che possono essere definiti \textit{ripetitivi}. 

Lo scopo della tesi è stato migliorare la rilevazione di oggetti all'interno di immagini termiche, per realizzare ciò è stata usata la rete neurale RetinaNet a cui sono state applicate alcune tecniche per ottenere prestazioni sempre maggiori. Dopo una prima fase di analisi e preprocessing dei dataset a nostra disposizione è inizia la fase di sperimentazione dove 
\end{document}